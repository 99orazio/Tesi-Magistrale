%\selectlanguage{italian}
\begin{abstract}
Nell'ambito delle comunicazioni e soprattutto dell'utilizzo di flussi di dati atti a fornire un servizio a diversi host, è molto importante riuscire a capire attraverso quali strumenti poter offrire il miglior rapporto tra \emph{performance} e \emph{sicurezza} nella comunicazione stessa. Lo scopo di questo lavoro è quello di poter creare un dispositivo di testing il quale permetta di poter mettere a confronto diversi protocolli VPN in modo da stabilire quale di esso si comporta meglio in determinate situazioni, in particolare nel fornire servizi i quali si basano sull'invio di \emph{flussi di dati} aventi precisi requisiti di \emph{QoS} che devono essere rispettati affinché non ci siano problemi nella comunicazione. Per raggiungere questo obiettivo sono stati posti in analisi diversi \emph{protocolli VPN} tra cui anche \emph{WireGaurd} versione \emph{Post Quantum} il quale è stato progettato per poter garantire un ottimo grado di sicurezza anche contro \emph{attaccanti} che potrebbero idealmente disporre di computer quantistici. Il dispositivo in questione ha come scopo secondario quello di poter fornire la possibilità di analizzare gli host presenti all'interno della rete locale \emph{LAN} così da poter recuperare importanti informazioni riguardanti essi; oltre a questo dovrà anche fornire un punto di accesso VPN in modo da consentire l'utilizzo delle risorse presenti nella rete \emph{LAN} del dispositivo anche da remoto, da parte degli host interessati.
\\[1cm]
\end{abstract} 
