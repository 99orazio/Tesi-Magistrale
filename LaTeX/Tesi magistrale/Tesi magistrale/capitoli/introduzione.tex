\phantomsection
%\addcontentsline{toc}{chapter}{Introduzione}
\chapter{Introduzione}
\markboth{Introduzione}{}
% [titolo ridotto se non ci dovesse stare] {titolo completo}

\begin{citazione}
Nel presente capitolo verrà effettuata una panoramica generale sulle problematiche trattate dal lavoro svolto, motivando la necessità della realizzazione del dispositivo in questione. Verrà successivamente presentata l'idea della soluzione proposta ad alto livello e del confronto prestazionale tra i diversi protocolli VPN posti in analisi.
\end{citazione}
\newpage

\section{Contesto ed obiettivo} { \setstretch{1.3}
Questo progetto mira a sviluppare un dispositivo di testbed \cite{testbed} il quale ha come obiettivo principale quello di poter eseguire una valutazione delle prestazioni su particolari flussi di traffico IP, impiegando VPN convenzionali e non in modo da poter estrapolare dai risultati, quale protocollo VPN si comporta meglio con i suddetti flussi. Il dispositivo realizzato avrà anche la capacità di eseguire una serie di scansioni per raccogliere una vasta gamma di informazioni sugli host connessi alla stessa rete locale (LAN). Tali informazioni saranno utili per chi gestisce la rete, fornendo una panoramica dettagliata degli host connessi. Inoltre, il dispositivo offrirà un servizio di connessione sicura realizzato mediante l’implementazione di un canale VPN (Virtual Private Network), che garantirà che tutte le comunicazioni passanti attraverso esso siano crittografate e sicure.

\subsection{Prestazioni}
Per quanto riguarda il confronto tra le diverse tecnologie VPN,  esso sarà incentrato in primo luogo sulla valutazione del livello di sicurezza offerto dalle diverse versioni dei protocolli in modo da stabilire quali offrono un grado maggiore di sicurezza alle utenze coninvolte; successivamente saranno poi presi in considerazione anche i parametri \emph{Bitrate} e \emph{Bandwidth} per stabilire quale protocollo VPN si comporta meglio in termini di trasferimento dati \cite{param}.

\section{Motivazioni}
Le motivazioni dietro questo progetto sono molteplici. In primis, con l’avvento dei computer quantistici, la crittografia post-quantistica sta diventando sempre più importante. Portando avanti una ricerca che coinvolge le prestazioni di WireGuard PQ, il progetto può contribuire a guidare lo sviluppo futuro di protocolli di sicurezza di rete, assicurando che siano pronti per l’era post-quantistica. In secondo luogo, la crescente complessità delle reti locali (LAN) ha reso sempre più difficile per gli amministratori di rete monitorare e gestire gli host connessi; la creazione di un dispositivo in grado di eseguire scansioni dettagliate e raccogliere informazioni sugli host può semplificare notevolmente questa gestione. Infine, la sicurezza delle comunicazioni è diventata una preoccupazione fondamentale nell’era digitale. Fornendo un servizio di connessione sicura attraverso un canale VPN, il dispositivo può garantire che tutte le comunicazioni siano crittografate e protette, aumentando così la sicurezza generale della rete. Queste sono le principali motivazioni che hanno guidato la realizzazione di questa testbed.

\section{Struttura della tesi}
La tesi è strutturata come segue:
\begin{itemize}
    \item \textbf{Capitolo 2 (Background):} questo capitolo fornisce una panoramica sui principali concetti indispensabili per comprendere a pieno le caratteristiche del dispositivo e dei protocolli VPN;
    \item \textbf{Capitolo 3 (Obiettivi della tesi):} questo capitolo descrive il cuore dello studio: stabilire gli obiettivi ci permetterà di delineare le funzionalità chiave che il sistema dovrà implementare per rispondere in modo efficace alle esigenze degli utenti finali e soprattutto stabilire quale protocollo VPN offre prestazioni adeguate in specifici casi;
    \item \textbf{Capitolo 4 (Progettazione del sistema):} In questo capitolo, trasformeremo i requisiti definiti nel capitolo precedente, in un progetto ben strutturato. Questo permetterà di costruire un sistema che non solo soddisfi le esigenze degli utenti, ma sia anche robusto, scalabile e manutenibile nel tempo;
    \item \textbf{Capitolo 5 (Realizzazione del sistema proposto):} In questo capitolo, le idee e i concetti, definiti precedentemente, saranno realizzate sotto forma di codice funzionante. Attraverso l’implementazione, il sistema prenderà forma, diventando un prodotto software pronto per essere utilizzato per raggiungere gli obiettivi prefissati;
    \item \textbf{Capitolo 6 (Testing e valutazione delle prestazioni del sistema):}  In questa sezione, il sistema sarà messo alla prova, misurando le sue prestazioni in termini di velocità, efficienza e affidabilità. Questo permetterà di individuare eventuali aree di miglioramento e di ottimizzare il sistema per garantire un buon grado di affidabilità;
    \item \textbf{Capitolo 7 (Conclusioni e sviluppi futuri):} In questa sezione, saranno riassunti i risultati ottenuti ed inoltre, saranno discussi i potenziali miglioramenti che potrebbero essere apportati in futuro.
\end{itemize}

}