\chapter{Background} %\label{1cap:spinta_laterale}
% [titolo ridotto se non ci dovesse stare] {titolo completo}
%
\begin{citazione}
Nell'ambito del presente capitolo sarà effettuata una panoramica sui principali concetti i quali rappresentano il punto cardine sul quale è stato sviluppato tutto il lavoro svolto.
\end{citazione}
\newpage

\section{Le reti informatiche}
Una rete informatica è un sistema di comunicazione che è diventato di fondamentale importanza in questa epoca. Essa permette di scambiare informazioni tra dispositivi, anche di diversa natura tra cui \emph{computer}, \emph{periferiche}, \emph{smartphone} ecc. Una rete informatica solitamente viene classificata in base alla sua dimensione geografica ed al mezzo trasmissivo che impiega per interconnettere i diversi dispositivi e scambiare informazioni.

\subsection{LAN}
La \emph{LAN} (Local Area Network) \cite{lan} è una piccola rete di dispositivi che si estende lungo una ristretta area geografica, come un edificio o una casa. Essa consente la condivisione di risorse tra cui file e stampanti, rendendo semplice la collaborazione tra gli utenti. Le LAN possono essere cablate, utilizzando cavi Ethernet e dispositivi di rete per collegare fisicamente i dispositivi come gli \emph{switch} ed i \emph{router}; o wireless, utilizzando segnali radio per trasmettere dati. Inoltre, le LAN possono essere connesse tra loro per formare una rete più ampia, che prende il nome di \emph{WAN} (Wide Area Network) \cite{wan}, permettendo la condivisione di risorse e la comunicazione tra dispositivi su una scala molto più ampia, spesso su base globale.

\section{Sicurezza e VPN}
La sicurezza è un aspetto fondamentale delle reti, in particolare delle LAN. Con l’aumento delle minacce alla sicurezza informatica, è fondamentale proteggere le informazioni sensibili che viaggiano attraverso la rete. Questo può includere l'impiego di diverse tecniche e strumenti tra cui l’utilizzo di \emph{firewall}, l’uso di software \emph{antivirus}, la \emph{crittografia dei dati}, l’implementazione di vari \emph{protocolli di sicurezza} per proteggere le informazioni in transito. Uno di questi protocolli è quello \emph{VPN} \cite{vpn}, o Virtual Private Network, che ha come obiettivo quello di creare un tunnel sicuro per il traffico di rete, proteggendo i dati da occhi indiscreti tramite il camuffamento dell'indirizzo IP reale dei dispositivi connessi.

\subsection{WireGuard VPN}
WireGuard \cite{WireGuard} è un esempio di un protocollo VPN che è noto per la sua velocità e semplicità. Offre crittografia di alto livello e ha un codice sorgente relativamente piccolo, il che facilita l’ispezione del codice per eventuali vulnerabilità. WireGuard è anche molto versatile, con supporto per vari sistemi operativi e tipi di hardware. Esso è progettato per essere facile da configurare e da utilizzare, rendendolo accessibile ad ogni tipo di utenza. Un aspetto unico di WireGuard è la sua implementazione della crittografia post-quantistica nella versione \emph{WireGuard PQ} \cite{WireGuardPQ}. Questo offre una protezione aggiuntiva contro le potenziali minacce dei computer quantistici, garantendo che le comunicazioni rimangano sicure anche nell’era post-quantistica. In sintesi, WireGuard è un protocollo VPN all’avanguardia che combina velocità, sicurezza e facilità d’uso, offrendo una soluzione efficace per la sicurezza delle reti informatiche.

\subsection{WireGuard Post Quantum}
La versione post quantistica di WireGuard \cite{WPQ} che è stata scelta per condurre i successivi esperimenti apporta importanti migliorie, in particolare alla fase di \emph{handshake} tramite il quale un qualsiasi \emph{client} ed un \emph{server}, possono reciprocamente autenticarsi in modo da poter stabilire un canale di comunicazione sicuro. Tale versione apporta delle modifiche al meccanismo di handshake impiegando l'algoritmo \emph{Kyber} appartenente alla famiglia degli algoritmi \emph{Crystals} \cite{crystal} in quanto risulta essere \emph{quantum resistant}. L'esecuzione di diversi test può contribuire a guidare lo sviluppo futuro dei protocolli di sicurezza di rete, assicurando che siano pronti per l’era post-quantistica. Queste sono le principali ragioni che hanno guidato la realizzazione di questo progetto.

\subsection{Introduzione alla crittografia Post Quantistica}
La crittografia post-quantistica \cite{crittografia} è un ramo della crittografia che si concentra sulla teoria e lo sviluppo di algoritmi crittografici resistenti agli attacchi da parte dei computer quantistici. Questo nuovo modello di computer è in grado di poter risolvere determinati problemi computazionali molto più velocemente rispetto ai computer classici grazie all'utilizzo dei \emph{qubit} invece dei \emph{bit} per rappresentare e manipolare le informazioni. A differenza dei bit classici, che possono essere \emph{0} o \emph{1}, i qubit possono esistere in uno stato di \emph{sovrapposizione}, dove possono essere sia 0 che 1 contemporaneamente ed è proprio grazie a questa proprietà che i computer quantistici possono risolvere alcuni problemi molto più velocemente dei loro modelli classici.
La realizzazione pratica di computer quantistici è ancora un’impresa difficile ma lo sviluppo della crittografia post-quantistica rimane comunque in pieno sviluppo, data la potenziale minaccia che i computer quantistici rappresentano per i sistemi crittografici attuali.

In sintesi, la sicurezza delle reti informatiche è un campo in continua evoluzione che richiede l’implementazione di protocolli di sicurezza robusti come VPN e WireGuard, nonché la preparazione per le future minacce della crittografia post-quantistica.

\section{Linux}
Linux è un sistema operativo \emph{open source} noto per la sua flessibilità e versatilità, con numerose distribuzioni disponibili per soddisfare una varietà di esigenze. Linux è composto da un componente essenziale chiamato \emph{Kernel} \cite{kernel}, il quale si interpone tra le risorse hardware ed il sistema operativo in modo da creare un'interfaccia di comunicazione . La filosofia alla base di Linux è quella di fornire un sistema operativo che sia completamente aperto e modificabile dall’utente, promuovendo la condivisione, la collaborazione e la libertà di scelta.

\subsection{Raspberry Pi}
Raspberry Pi è una piccola scheda \emph{On Board} di dimensioni molto ridotte la quale è stata progettata principalmente per uso didattico. Essa permette l'esecuzione di numerose applicazioni e di diversi sistemi operativi grazie al tipo di architettura \emph{ARM} con la quale queste schede sono state ideate \cite{arm}. Con il tempo sono state prodotte diverse schede, ognuna delle quali rappresenta l'evoluzione della precedente in termini di prestazioni. Attualmente questi dispositivi sono impiegati in diversi settori tra cui quello delle \emph{telecomunicazioni}, \emph{IoT}, \emph{networking}, grazie al costo relativamente contenuto ed al set di periferiche disponibili.

\section{Tecnologie utilizzate}
\subsection{Docker}
Docker è un software \emph{open source} impiegato per consentire l’esecuzione e la gestione di software ospitati all’interno di \emph{container} autonomi, eseguibili sia in locale che in cloud \cite{cont}. Questo strumento viene spesso impiegato quando si vuole avere un container contenente tutto il necessario per l’esecuzione di un’applicazione come: \emph{librerie}, \emph{codice}, \emph{ambiente di esecuzione} ecc.
Nello specifico è stata utilizzata la versione \emph{Compose} la quale offre la possibilità di poter gestire un’applicazione utilizzando più di un unico container così da poter aver la possibilità di estendere il progetto in futuro, in caso di necessità \cite{compose}.

\subsection{Celery}
Celery è un \emph{task scheduler} che permette di eseguire lavori in modalità \emph{asincrona}, utile quando il compito da svolgere può richiedere diverso tempo e non si ha la possibilità di attendere la sua fine. Celery fornisce anche una \emph{API Python} in modo da poter definire un'attività da svolgere e soprattutto gestire il loro stato. Esso è molto flessibile ed offre la possibilità di poter utilizzare diversi \emph{message broker} come ad esempio \emph{Redis} o \emph{RabbitMQ} \cite{broker}.

\subsection{Redis}
Redis è un \emph{database} di tipo \emph{chiave-valore} che offre tempi di risposta inferiori al millisecondo per cui viene molto spesso impiegato dato che gode di ottime prestazioni in lettura e scrittura dei dati \cite{kv}. Oltre ad essere impiegato come database, Redis ha la possibilità di funzionare anche come \emph{cache} o \emph{message broker} ed è per questo motivo che è stato scelto per essere utilizzato con Celery. Un message broker è un applicazione che svolge il compito di intermediario tra due diverse architetture che comunicano tramite messaggi.

\subsection{Nmap}
Nmap è uno scanner di rete il quale permette di ricostruire la struttura della rete interna. Esso viene ampiamente utilizzato dato che fornisce la possibilità di personalizzare il tipo di scansione da eseguire; alcune delle varie features che caratterizzano Nmap sono:
\begin{itemize}
    \item Individuare i diversi dispositivi in rete;
    \item Identificare i servizi in esecuzione sui dispositivi;
    \item Rilevare le versioni dei servizi in esecuzione;
    \item Individuare il sistema operativo in esecuzione sui dispositivi.
\end{itemize}

\subsection{Flask}
Flask è un framework scritto in \emph{Python} il quale permette lo sviluppo backend di un'applicazione web. Ultimamente sta prendendo piede grazie alla sua semplicità ed alla possibilità di scalare facilmente per la costruzione di applicazioni complesse; un'altra caratteristica importante riguarda la comunity che offre costantemente la possibilità di poter integrare nuove funzionalità.
%\section{Lavori correlati}

\section{Metodologia di testing}
Per quanto riguarda i test che sono stati eseguiti con la testbed \emph{TurtleVPN}, essi sono stati ideati, con lo scopo di simulare nel modo più fedele possibile, diversi flussi di dati aventi specifici requisiti \emph{time sensitive} in modo da stabilire le performance dei diversi protocolli VPN presi in esame. 
\subsection{iPerf}
Il tool in questione è un particolare strumento di diagnostica di rete Open Source il quale consente di misurare le performance della rete tramite la regolazione di numerosi parametri che consentono di simulare in modo fedele i diversi tipi di flussi di dati che possono scambiarsi due differenti host in rete. Per poter eseguire i test è necessario che i due host installino entrambi il tool in modo da poter stabilire un collegamento e poter proseguire con il trasferimento dei flussi ideati.
